\UseRawInputEncoding
%
% Module 3 Chapter 6 Homework
% CSC160-C00: Computer Science I (C++)
% Author: Ashton Hellwig
% Date: 02 April 2020
%


\documentclass[a4paper, 11pt]{article}
  % Packages
  \usepackage[latin1,utf8]{inputenc}  % Encoding
  \usepackage[english]{babel}         % Internationalization
  \usepackage{times}                  % Times New Roman font
  \usepackage{soul}                   % Highlighting
  \usepackage{hyperref}               % Links (internal and external)
  \usepackage{fancyhdr}               % Headers and footers
  \usepackage[dvipsnames]{xcolor}     % Text Colors
  \usepackage{listings}               % Code Snippets
  \usepackage[section]{algorithm}     % For TOC support
  \usepackage{algpseudocode}          % Algorithmic notation environments
  \usepackage{enumitem}               % Ordered lists
  \usepackage{geometry}               % Page layout
  \usepackage{graphicx}               % Image support
  \usepackage[toc, page]{appendix}    % Appendix
  \usepackage{setspace}               % Paragraph and line spacing
  \usepackage{bookmark}               % Required for appendix
  \usepackage{adjustbox}              % Required for appendix
  \usepackage{csquotes}               % Required for appendix
  \usepackage{amsthm}                 % Theorem environments
  \usepackage{array}                  % Arrays
  \usepackage{makecell}               % Table helpers
  \usepackage{amsmath}                % Mathematical symbols
  \usepackage[fleqn]{mathtools}       % Mathematical environments
  \usepackage{amssymb}                % Misc. symbols for logic and math
  \usepackage{relsize}                % Relative Sizing
  \usepackage{multicol}               % Multi-figure displays (grid)
  \usepackage{etoolbox,refcount}      % Required for MDFramed
  \usepackage{parcolumns}             % Paragraph grids
  \usepackage{mdframed}               % Colored box environments
  \usepackage{float}                  % Floating Environments 
  \usepackage{aliascnt}
  \usepackage[                        % Bibliography management
    backend=biber,%
    style=apa%
  ]{biblatex}
  \usepackage{mathtools}
  \usepackage{empheq}
  \usepackage[skins,theorems]{tcolorbox}

  % Bibliography Setup
  \addbibresource{main.bib}
  \newcommand{\CiteSection}[2]{%
    (\autocite{#1}, ~\S {#1})%
  }

%   \UseRawInputEncoding

  % Tables
  \renewcommand\theadalign{bc}
  \renewcommand\theadfont{\bfseries}
  \renewcommand\theadgape{\Gape[4pt]}
  \renewcommand\cellgape{\Gape[4pt]}

  % Lists
  \newcounter{countitems}
  \newcounter{nextitemizecount}
  \newcommand{\setupcountitems}{%
    \stepcounter{nextitemizecount}%
    \setcounter{countitems}{0}%
    \preto\item{\stepcounter{countitems}}%
  }
  \makeatletter
  \newcommand{\computecountitems}{%
    \edef\@currentlabel{\number\c@countitems}%
    \label{countitems@\number\numexpr\value{nextitemizecount}-1\relax}%
  }
  \newcommand{\nextitemizecount}{%
    \getrefnumber{countitems@\number\c@nextitemizecount}%
  }
  \newcommand{\previtemizecount}{%
    \getrefnumber{countitems@\number\numexpr\value{nextitemizecount}-1\relax}%
  }
  \makeatother
  \newenvironment{AutoMultiColItemize}{%
  \ifnumcomp{\nextitemizecount}{>}{3}{\begin{multicols}{2}}{}%
  \setupcountitems\begin{itemize}}%
  {\end{itemize}%
  \unskip\computecountitems\ifnumcomp{\previtemizecount}{>}{3}{\end{multicols}}{}}



  % Colors
  \newcommand{\commentstylecolor}{\color{Gray}}
  \newcommand{\keywordstylecolor}{\color{MidnightBlue}}
  \newcommand{\stringstylecolor}{\color{ForestGreen}}
  \newcommand{\questioninput}{\color{Red}}
  \newcommand{\answertcolor}{\color{Green}}
  \newcommand{\myanswer}{\answertcolor{\hl}}

  % Symbols
  \newcommand{\answerflow}{\rotatebox[origin=c]{180}{$\Lsh$}}
  \newcommand{\toanswer}{\mathlarger{\mathlarger{\answerflow}}\quad}

  % Math
  \newcommand{\highlight}[1]{%
    \colorbox{green!50}{$\displaystyle#1$}}

  % Image Directory
  \graphicspath{ {screenshots/} }


  % Hyperlink Setup
  \hypersetup{
    colorlinks = true,
    urlcolor = blue,
    linkcolor = blue
  }


  % Algorithm Settings
  \newcommand{\pluseq}{\mathrel{{+}{=}}}
  \newcommand{\minuseq}{\mathrel{{-}{=}}}


  % Syntax-Highlighting for Code Snippets
  \lstset{
    backgroundcolor=\color{white},
    breaklines=true,%
    captionpos=b,%
    frame=tlrb,%
    columns=fixed,%
    tabsize=2,%
    numbers=left,%
    showstringspaces=false,%
    commentstyle=\commentstylecolor,%
    keywordstyle=\keywordstylecolor,%
    stringstyle=\stringstylecolor%
  }
  \lstset{literate=
  {á}{{\'a}}1 {é}{{\'e}}1 {í}{{\'i}}1 {ó}{{\'o}}1 {ú}{{\'u}}1
  {Á}{{\'A}}1 {É}{{\'E}}1 {Í}{{\'I}}1 {Ó}{{\'O}}1 {Ú}{{\'U}}1
  {à}{{\`a}}1 {è}{{\`e}}1 {ì}{{\`i}}1 {ò}{{\`o}}1 {ù}{{\`u}}1
  {À}{{\`A}}1 {È}{{\'E}}1 {Ì}{{\`I}}1 {Ò}{{\`O}}1 {Ù}{{\`U}}1
  {ä}{{\"a}}1 {ë}{{\"e}}1 {ï}{{\"i}}1 {ö}{{\"o}}1 {ü}{{\"u}}1
  {Ä}{{\"A}}1 {Ë}{{\"E}}1 {Ï}{{\"I}}1 {Ö}{{\"O}}1 {Ü}{{\"U}}1
  {â}{{\^a}}1 {ê}{{\^e}}1 {î}{{\^i}}1 {ô}{{\^o}}1 {û}{{\^u}}1
  {Â}{{\^A}}1 {Ê}{{\^E}}1 {Î}{{\^I}}1 {Ô}{{\^O}}1 {Û}{{\^U}}1
  {œ}{{\oe}}1 {Œ}{{\OE}}1 {æ}{{\ae}}1 {Æ}{{\AE}}1 {ß}{{\ss}}1
  {ű}{{\H{u}}}1 {Ű}{{\H{U}}}1 {ő}{{\H{o}}}1 {Ő}{{\H{O}}}1
  {ç}{{\c c}}1 {Ç}{{\c C}}1 {ø}{{\o}}1 {å}{{\r a}}1 {Å}{{\r A}}1
  {€}{{\euro}}1 {£}{{\pounds}}1 {«}{{\guillemotleft}}1
  {»}{{\guillemotright}}1 {ñ}{{\~n}}1 {Ñ}{{\~N}}1 {¿}{{?`}}1
}
  \newenvironment{alltt}{\ttfamily}{\par}
  \lstMakeShortInline[language=c++,columns=fixed]|

  % Page Configuration
  %% Style
  \pagestyle{fancy}

  %% Layout
  \geometry{%
    a4paper,%
    top=2.5cm,%
    bottom=2.5cm,%
    left=2.5cm,%
    right=2.5cm%
  }
  \setlength{\headheight}{15pt}
  \setlength{\floatsep}{9pt}
  \setlength{\parindent}{2em}
  \setlength{\parskip}{0.3em}
  \renewcommand{\baselinestretch}{0.75}

  %% Title page
  \title{Module 3 Chapter 6 Homework}
  \author{Ashton Hellwig}
  \date\today
  \setcounter{tocdepth}{3}

  % MDFramed Config
  \mdfdefinestyle{AnswerFrame}{%
    linecolor=black,
    outerlinewidth=2pt,
    roundcorner=20pt,
    innertopmargin=4pt,%
    innerbottommargin=4pt,%
    innerrightmargin=4pt,%
    innerleftmargin=4pt,%
    leftmargin = 4pt,%
    rightmargin = 4pt,%
    backgroundcolor=green!20%
  }


   % Floating Equations
  \newaliascnt{eqfloat}{equation}
  \newfloat{eqfloat}{h}{eqflts}
  \floatname{eqfloat}{Equation}

  \newcommand*{\ORGeqfloat}{}
  \let\ORGeqfloat\eqfloat
  \def\eqfloat{%
    \let\ORIGINALcaption\caption
    \def\caption{%
      \addtocounter{equation}{-1}%
      \ORIGINALcaption
    }%
    \ORGeqfloat
  }

%   Question 1
%   \section{Question}
%     Placeholder:
%     \begin{enumerate}[label=\alph*.]
%       \item Placeholder.
%     \end{enumerate}
%     % Question 1 Solution
%     \subsection{Solution}
%       \begin{enumerate}[label=\alph*.]
%           \item Placeholder.
%       \end{enumerate}



% Document Content
\begin{document}
  % Title Page
  \maketitle
  \tableofcontents
  \lstlistoflistings
  \newpage

  % Question 1
  \section{Question 1}
    Determine the value of each of the following expressions:
    \begin{enumerate}[label=\Alph*.]
      \item $|static_cast<char>(toupper('7'))|$
      \item $|static_cast<char> (toupper('@'))|$
      \item $|static_cast<char> (toupper('s'))|$
      \item $|static_cast<char> (toupper('J'))|$
      \item $|static_cast<char> (tolower('*'))|$
      \item $|static_cast<char> (tolower(';'))|$
      \item $|static_cast<char> (tolower('w'))|$
      \item $|static_cast<char> (tolower('('))|$
    \end{enumerate}

    % Question 1 Solution
    \subsection{Solution}
      \begin{enumerate}[label=\alph*.]
        \begin{minipage}{0.45\textwidth}
          \item \begin{mdframed}[style=AnswerFrame]
            Placeholder.
            \end{mdframed}
          \item \begin{mdframed}[style=AnswerFrame]
            Placeholder.
            \end{mdframed}
        \end{minipage}\hfill
        \begin{minipage}{0.45\textwidth}
          \item \begin{mdframed}[style=AnswerFrame]
            Placeholder.
            \end{mdframed}
          \item \begin{mdframed}[style=AnswerFrame]
            Placeholder.
            \end{mdframed}
        \end{minipage}\hfill
        \begin{minipage}{0.45\textwidth}
          \item \begin{mdframed}[style=AnswerFrame]
            Placeholder.
            \end{mdframed}
          \item \begin{mdframed}[style=AnswerFrame]
            Placeholder.
            \end{mdframed}
        \end{minipage}\hfill
        \begin{minipage}{0.45\textwidth}
          \item \begin{mdframed}[style=AnswerFrame]
            Placeholder.
            \end{mdframed}
          \item \begin{mdframed}[style=AnswerFrame]
            Placeholder.
            \end{mdframed}
        \end{minipage}
      \end{enumerate}
      
      
  \section{Question 2}
    Consider the following function
    
    \begin{lstlisting}[language=c++,caption={Question 1 Problem}]
int mystery(int x, double y, char ch) {
  if (x == 0 && ch > 'A')
    return(static_cast<int>(pow(y, 2)) + static_cast<int> (ch));
  else if (x > 0)
    return(x + static_cast<int>(sqrt(y)) - static_cast<int> (ch));
  else
    return(2 * x + static_cast<int>(y) - static_cast<int> (ch));
}
    \end{lstlisting}
    
    What is the output of the following statements?
    
    \begin{enumerate}[label=\Alph*.]
      \item $|cout << mystery(0, 6.5, 'K') << endl;|$
      \item $|cout << mystery(4, 16.0, '#') << endl;|$
      \item $|cout << 2 * mystery(-11, 13.8, '8') << endl;|$
    \end{enumerate}
    
      % Question 2 Solution
      \subsection{Solution}
        \begin{enumerate}[label=\alph*.]
          \begin{minipage}{0.45\textwidth}
            \item \begin{mdframed}[style=AnswerFrame]
              Placeholder.
              \end{mdframed}
            \item \begin{mdframed}[style=AnswerFrame]
              Placeholder.
              \end{mdframed}
          \end{minipage}\hfill
          \begin{minipage}{.45\textwidth}
          \item \begin{mdframed}[style=AnswerFrame]
            Placeholder.
            \end{mdframed}
        \end{minipage}
    \end{enumerate}
    
    
  \newpage
  \section{Question 3}
    Consider the following program:

    \begin{lstlisting}[language=c++,caption={Question 3 Problem}]
#include <iostream>
using namespace std;

void func1();
void func2();

int main() {
  int num;

  cout << "Enter 1 or 2: ";
  cin >> num;
  cout << endl;

  cout << "Take ";

  if (num == 1)
    func1();
  else if (num == 2)
    func2();
  else
    cout << "Invalid input. You must enter a 1 or 2" << endl;
  return 0;
}

void func1() {
cout << "Programming I." <<endl;
}

void func2() {
cout << "Programming II." <<endl;
}
    \end{lstlisting}

    \begin{enumerate}[label=\Alph*.]
      \item What is the output if the input is \texttt{1}?
      \item What is the output if the input is \texttt{2}?
      \item What is the output if the input is \texttt{3}?
      \item What is the output if the input is \texttt{-1}?
    \end{enumerate}
    
    % Question 3 Solution
    \subsection{Solution}
      \begin{enumerate}[label=\alph*.]
        \begin{minipage}{0.45\textwidth}
          \item \begin{mdframed}[style=AnswerFrame]
            Placeholder.
            \end{mdframed}
          \item \begin{mdframed}[style=AnswerFrame]
            Placeholder.
            \end{mdframed}
        \end{minipage}\hfill
        \begin{minipage}{.45\textwidth}
          \item \begin{mdframed}[style=AnswerFrame]
            Placeholder.
            \end{mdframed}
          \item \begin{mdframed}[style=AnswerFrame]
            Placeholder.
            \end{mdframed}
        \end{minipage}
      \end{enumerate}


  % Question 4
  \newpage
  \section{Question 4}
    Consider the following program:

    \begin{lstlisting}[language=c++,caption={Question 4 Problem}]
#include <iostream>
#include <cmath>
#include<iomanip>

using namespace std;

void traceMe(double x, double y);

int main() {
  double one, two;

  cout << "Enter two numbers: ";
  cin >> one >> two;
  cout << endl;

  traceMe(one, two);
  traceMe(two, one);
  return 0;
}

void traceMe(double x, double y) {
  double z;

  if (x != 0)
    z = sqrt(y) / x;
  else {
    cout << "Enter a nonzero number: ";
    cin >> x;
    cout << endl;
    z = floor(pow(y, x));
  }
  cout << fixed << showpoint << setprecision(2);
  cout << x << ", " << y << ", " << z << endl;
}
    \end{lstlisting}

    \begin{enumerate}[label=\Alph*.]
      \item What is the output if the input is \texttt{3 625}?
      \item What is the output if the input is \texttt{24 1024}?
      \item What is the output if the input is \texttt{0 196}?
    \end{enumerate}

    % Question 4 Solution
    \subsection{Solution}
    \begin{enumerate}[label=\alph*.]
      \begin{minipage}{0.45\textwidth}
        \item \begin{mdframed}[style=AnswerFrame]
          Placeholder.
          \end{mdframed}
        \item \begin{mdframed}[style=AnswerFrame]
          Placeholder.
          \end{mdframed}
      \end{minipage}\hfill
      \begin{minipage}{.45\textwidth}
        \item \begin{mdframed}[style=AnswerFrame]
          Placeholder.
          \end{mdframed}
      \end{minipage}
    \end{enumerate}


  % Question 5
  \newpage
  \section{Question 5}
    Consider the following function definition:

    \begin{lstlisting}[language=c++,caption={Question 5 Problem}]
void defaultParam(int num1, int num2 = 7, double z = 2.5) {
  int num3;

  num1 = num1 + static_cast<int>(z);
  z = num2 + num1 * z;
  num3 = num2 - num1;
  cout << "num3 = " << num3 << endl;
}
    \end{lstlisting}

    What is the output of the following function calls?

    \begin{figure}[H]
      \centering
      \begin{tabular}{|l|l|}
        \hline\\
        a & $|defaultParam(7);|$\\
        \hline\\
        b & $|defaultParam(8, 2);|$\\
        \hline\\
        c & $|defaultParam(0, 1, 7.5);|$\\
        \hline\\
        d & $|defaultParam(1, 2, 3.0);|$ \\
        \hline
      \end{tabular}
    \end{figure}

    % Question 5 Solution
    \subsection{Solution}
    \begin{enumerate}[label=\alph*.]
      \begin{minipage}{0.45\textwidth}
        \item \begin{mdframed}[style=AnswerFrame]
          Placeholder.
          \end{mdframed}
        \item \begin{mdframed}[style=AnswerFrame]
          Placeholder.
          \end{mdframed}
      \end{minipage}\hfill
      \begin{minipage}{.45\textwidth}
        \item \begin{mdframed}[style=AnswerFrame]
          Placeholder.
          \end{mdframed}
        \item \begin{mdframed}[style=AnswerFrame]
          Placeholder.
          \end{mdframed}
      \end{minipage}
    \end{enumerate}


  % Bibliography
  \newpage
  \nocite{textbook}
  \printbibliography[%
    heading=bibintoc,%
    title={Works Consulted}%
  ]{}
  
%   Appendix
%   \newpage
%   \appendix
%   \section{Question 1}
%     Placeholder.
\end{document}
